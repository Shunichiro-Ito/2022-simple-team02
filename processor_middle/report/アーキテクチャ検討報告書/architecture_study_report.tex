\documentclass[a4j,titlepage]{jarticle}
\usepackage[dvipdfmx]{graphicx}
\usepackage{ascmac}
\usepackage{float}
\usepackage{amssymb}%にやりイコールを使う
\usepackage{multirow}
\usepackage{multicol}
%\usepackage{color}

\begin{document}

\title{2022 年度 3 回生前期学生実験 HW  \\ \bf team02 アーキテクチャ検討報告書}
% ↓ここに自分の氏名を記入
\author{アーキテクチャ検討報告書作成者:植田健斗\\
グループメンバー:\\伊藤舜一郎 (学籍番号:1029-32-7548)
\\植田健斗 (学籍番号:1029-32-6498)}
\西暦
\date{提出期限:5月12日18時 提出日: \today} % コンパイル時の日付が自動で挿入される
\maketitle
\newpage

\section{要求仕様,設計目標,設計方針,特長}

\subsection{実現する機能}
\subsection{性能の目標数値}

\section{高速化/並列処理の方式}

\subsection{拡張命令}
\subsection{動作周波数}
\subsection{パイプライン化}
\subsection{並列化}


\section{性能/コストの予測}

\subsection{SIMPLE/Bに比べて性能やハードウェア量が何倍程度か,それは妥当な見積もりか}
\subsection{ソート速度コンテストでの計算時間,サイクル数の予測}

\section{考察}

\end{document}